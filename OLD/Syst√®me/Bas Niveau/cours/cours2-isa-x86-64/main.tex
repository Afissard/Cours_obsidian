\documentclass[xcolor=table,svgnames,10pt,french]{beamer}

\usepackage{fontspec-luatex}
\usepackage{polyglossia}
\usepackage{listings}
\usepackage{multirow}
\usepackage{textcomp}
\usepackage{tikz}
\usepackage{hyperref}
\usepackage{hhline}
\usepackage{graphicx}
\usepackage{url}
\usepackage{amsmath,amssymb}
\usepackage{algorithm}
\usepackage{algpseudocode}
\usepackage{tabularx}
\usepackage{soul}
\usepackage[binary-units]{siunitx}
\usepackage[cache=false]{minted}
\usepackage{tcolorbox}

\usetikzlibrary{arrows,automata,backgrounds,shapes,snakes,patterns,decorations, shapes.arrows, positioning}

\setmainlanguage{french}

\usetheme{metropolis}

\definecolor{maDarkBrown}{HTML}{604c38}
\definecolor{maDarkTeal}{HTML}{23373b}
\definecolor{maLightBrown}{HTML}{EB811B}
\definecolor{maLightGreen}{HTML}{14B03D}

\newenvironment<>{remarque}[1][\undefined]{%
  \begin{actionenv}#2%
    \ifx#1\undefined%
    \def\insertblocktitle{Remarque}%
    \else%
    \def\insertblocktitle{Remarque ({\em#1})}%
    \fi%
    \par%
    \mode<presentation>{%
      \setbeamercolor{block title}{fg=black,bg=maLightGreen!50}
      \setbeamercolor{block body}{fg=black,bg=maLightGreen!20}
    }%
    \usebeamertemplate{block begin}\em}
  {\par\usebeamertemplate{block end}\end{actionenv}}



\newcommand{\demo}[0]{%
  \hfill\begin{tcolorbox}[hbox,colback=maLightGreen!50,colframe=maLightGreen,halign=flushleft]\bf → Démo\end{tcolorbox}}

\hypersetup{
  colorlinks=true
}

\title[R2.04]{Ressource R2.04 \\ Communication et fonctionnement bas niveau}
\subtitle{Cours n°2 \\Le jeu d'instruction x86-64}
\author[]{Sébastien Faucou}
\date{version du \today}

\AtBeginSection[]
{
  \frame{
    \frametitle{Plan du cours}
    \tableofcontents[currentsection, sectionstyle=show/shaded,subsectionstyle=show/show/hide]
  }
}

\AtBeginSubsection[]
{
  \frame{
    \frametitle{Plan du cours}
    \tableofcontents[currentsection, currentsubsection, subsectionstyle=show/shaded/hide]
  }
}


\AtBeginSubsubsection[]
{
  \frame{
    \frametitle{Plan du cours}
    \tableofcontents[currentsection, currentsubsection, subsubsectionstyle=show/shaded/hide/hide]
  }
}

\begin{document}

\frame{
  \maketitle
}

\section{Introduction}

\begin{frame}{le jeu d'instruction x86-64}
   
    \begin{block}{x86-64}
        \begin{itemize}
            \item Version 64 bits de l'ISA x86, proposée par AMD en 1999, compatible avec les précédentes incarnations 8, 16 et 32 bits de l'ISA Intel.
                \begin{itemize}
                    \item De son côté, Intel et HP avaient proposées l'ISA IA-64, incompatible, ce qui a limité fortement sa diffusion.
                \end{itemize}
            \item Fait passer un cap à l'ISA x86 : adresses sur 64 bits, augmentation du nombre de registres, meilleur support du calcul flottant, ajout d'instructions vectorielles, etc.
        \end{itemize}
    \end{block}

    Aujourd'hui, c'est le jeu d'instruction principal des processeurs Intel et AMD.

    À noter que l'ISA la plus utilisée dans le monde aujourd'hui est sans doute l'ISA A64 de ARM.

\end{frame}

\begin{frame}{Anatomie d'une instruction}

    Comme dans tous les jeux d'instruction, une instruction x86-64est composée (entre autre)
    \begin{itemize}
        \item d'un \alert{opcode} : identifie l'opération

            Ex : push, mov, add, call, ret

            \medskip

        \item d'un \alert{format} : indique la taille des données manipulées

            Pour les types entiers : b, w, l, q

            \medskip

        \item d'\alert{opérandes} sources et/ou destinations : constantes, registres, emplacements en mémoire

            Ex : \%rbp, \$16, 8(\%rsp)

    \end{itemize}
\end{frame}

\section{Les registres et la mémoire}

\subsection{Les registres}

\begin{frame}

    \begin{block}{Fichiers de registres}
        Mémoire de quelques centaines d'octets, située dans le processeur, et connectée aux unités fonctionnelles.
        Adressable par mot mémoire, chaque mot disposant d'un nom spécifié dans le jeu d'instruction.
    \end{block}

    L'ISA x86-64 spécifie :
    \begin{itemize}
        \item 16 registres  de 64 bits pour le calcul sur les entiers
        \item 16 pseudos-registres  de 32 bits pour le calcul sur les entiers\\ → 4 octets de poids faible des registres 64 bits 
        \item 8 registres de 8 bits pour la compatibilité 8086 → \\ 2 octets de poids faible de certains pseudo-registres 32 bits
        \item des registres de contrôle ou description de l'état de la machine\\ ne peuvent être utilisés comme source ou cible des instructions
    \end{itemize}
\end{frame}

\begin{frame}
    \frametitle{Noms des registres x86-64}
    \begin{columns}[T]
        \begin{column}{.5\textwidth}
        \begin{itemize}
            \item \texttt{\%rax} (\texttt{\%eax}, \texttt{\%ah}, \texttt{\%al})
            \item \texttt{\%rbx} (\texttt{\%ebx}, \texttt{\%bh}, \texttt{\%bl})
            \item \texttt{\%rcx} (\texttt{\%ecx}, \texttt{\%ch}, \texttt{\%cl})
            \item \texttt{\%rdx} (\texttt{\%edx}, \texttt{\%dh}, \texttt{\%dl})
            \item \texttt{\%rsi} (\texttt{\%esi})
            \item \texttt{\%rdi} (\texttt{\%edi})
            \item \texttt{\%rsp} (\texttt{\%esp}, pointeur de pile)
            \item \texttt{\%rbp} (\texttt{\%ebp}, pointeur de base )
        \end{itemize}    
        \end{column}
        \begin{column}{.5\textwidth}
           \begin{itemize}
               \item \texttt{\%r8} (\texttt{\%r8d})
               \item \texttt{\%r9} (\texttt{\%r9d})
               \item \texttt{\%r10} (\texttt{\%r10d})
               \item \texttt{\%r11} (\texttt{\%r11d})
               \item \texttt{\%r12} (\texttt{\%r12d})
               \item \texttt{\%r13} (\texttt{\%r13d})
               \item \texttt{\%r14} (\texttt{\%r14d})
               \item \texttt{\%r15} (\texttt{\%r15d})
           \end{itemize} 
        \end{column}
    \end{columns}

    Les micro-architectures modernes ont bien plus de registres et incluent un étage de renommage pour que les programmes accèdent aux registres physiques.
\end{frame}

\subsection{La mémoire et les modes d'adressage}

\begin{frame}
    \frametitle{L'espace d'adresage}
    
    L'espace d'adressage fait $2^{64}$ octets :  $2^{61}$ mots de 64 bits (8 octets).
    C'est un espace \alert{virtuel} : la machine n'a pas besoin d'avoir autant de mémoire.

    \medskip

    Les micro-architectures actuelles utilisent seulement les 48 bits de poids faibles, soit 256 tébioctets adressables.

    \medskip

    Les adresses utilisées dans le programme binaire sont virtuelles.
    L'association entre adresses virtuelles et adresses phyisiques est faite \alert{à l'exécution par le système de mémoire virtuelle} (collaboration système d'exploitation / micro-architecture).

\end{frame}

\begin{frame}
    \frametitle{Modes d'adressage}

    Une adresse mémoire peut être utilisée comme opérande source ou cible d'une instruction.

    \medskip

    Plusieurs options possibles :
    \begin{itemize}
        \item \alert{adresse absolue} : objets alloués statiquements (code, globales, constantes)
        \item \alert{adresse relative} au pointeur de pile : objets alloués automatiquement (variables locales)
        \item \alert{adresse relative} à une adresse contenue en mémoire : objets allouées dynamiquements (variables dynamiques, certaines variables locales)
    \end{itemize}

    Adresse contenue en mémoire = pointeur !
\end{frame}

\begin{frame}{Les modes d'adressage}

    Soient $ v \in [0,2^{64}], r, r1, r2 \in Regs, e \in \{1, 2, 4, 8\}$. $M$ (resp. $R$) est un tableau qui représente la mémoire (resp. le fichier de registres).

    \begin{center}
        \begin{tabular}{lll}
            \hline
            Syntaxe & Sémantique & Mode d'adressage \\
            \hline
            \hline
             {\tt v} & $M[v]$ & Absolu \\
             {\tt (r)} & $M[R[r]]$ & Indirect \\
             {\tt v(r)} & $M[v + R[r]]$ & Base + déplacement\\
             {\tt (r1, r2)} & $M[R[r_1] + R[r_2]]$ & Indexé \\
             {\tt v(r1, r2)} & $M[v + R[r_1] + R[r_2]]$ & Indexé \\
             {\tt (,r,e)} & $M[R[r] \times e]$ & Indexé avec échelle \\
             {\tt v(,r,e)} & $M[v + (R[r] \times e)]$ & Indexé avec échelle \\
             {\tt (r1,r2,e)} & $M[R[r_1] + (R[r_2] \times e)]$ & Indexé avec échelle \\
             {\tt v(r1,r2,e)} & $M[v+ R[r_1] + (R[r_2] \times e)]$ & Indexé avec échelle \\
            \hline
        \end{tabular}
    \end{center}
    Note : {\tt \$v} désigne la constante entière \texttt{v}.
\end{frame}

\begin{frame}{Modes d'adressage : exemples}

Soit l'état partiel de la machine :

\medskip

\begin{columns}
    \begin{column}{0.5\textwidth}
        \centering
        $R$ \\
        \smallskip
        \begin{tabular}{c|c|}
            \hhline{~-}
            \texttt{\%rax} & \texttt{0xff \ldots ffab10} \\
            \hhline{~-}
            \texttt{\%rdx} & \texttt{0x4} \\
            \hhline{~-}
        \end{tabular}
    \end{column}
    \begin{column}{0.5\textwidth}
        \centering
        $M$ \\
        \smallskip
        \begin{tabular}{c|c|}
            \hhline{~-}
            \texttt{0xff \ldots ffab10} & \texttt{0x1} \\
            \hhline{~-}
            \texttt{0xff \ldots ffab18} & \texttt{0x10} \\
            \hhline{~-}
            \texttt{0xff \ldots ffab20} & \texttt{0x100} \\
            \hhline{~-}
            \texttt{0xff \ldots ffab28} & \texttt{0x1000} \\
            \hhline{~-}
            \texttt{0xff \ldots ffab30} & \texttt{0x2} \\
            \hhline{~-}
            \texttt{0xff \ldots ffab38} & \texttt{0x20} \\
            \hhline{~-}
        \end{tabular}

    \end{column}
\end{columns}

\bigskip

Complétez le tableau :

\begin{center}
    \begin{tabular}{|lll|}
        \hline
        Expression &
        Sémantique &
        Valeur
        \\
        \hline
        \hline

        \texttt{(\%rax)} &
        $M[R[\%rax]]$ &
        1
        \\
        \hline

        \texttt{0x10(\%rax)}
        & $M[R[\%rax]+0x10]$
        &
        \\
        \hline

        \texttt{(\%rax, \%rdx)}
        &
        &
        \\
        \hline

        & $M[R[\%rax] + 2 \times R[\%rdx]]$
        &
        \\
        \hline
    \end{tabular}
\end{center}

\end{frame}

\subsection{Déplacer des données}

\begin{frame}
    \frametitle{Placement des données en mémoire}

    1) Les objets du programmes sont en mémoire.

    Objet = variables, fonctions.

    Le nom donné à l'objet dans le code source est appelé symbole.
    Chaque symbole est associée à une adresse pour toute sa durée de vie :
    \begin{itemize}
        \item à l'édition des liens : symboles statiquements,
        \item à l'exécution : symboles automatiques et dynamiques.
    \end{itemize}

    \bigskip

    2) Même avec une ISA CISC, la majorité des opérations agissent uniquement sur les registres.

\end{frame}

\begin{frame}[fragile]
    \frametitle{Les opérations de copie}

    1) + 2) $\Longrightarrow$ les programmes doivent sans cesse copier les données entre mémoire et registres
    \begin{itemize}
        \item un compilateur optimisant minimiser ces mouvements : propagation de constante, association de symbole à des registres, etc.
    \end{itemize}

    \bigskip

    \begin{itemize}
        \item \verb|mov src, dst| entre mémoire et registre
        \item \verb|push reg| ou \verb|pop reg| entre pile et registres
        \item \verb|push reg| correspond à \verb|sub $8, %rbp| suivi de \verb|mov reg, (%rsp)|
    \end{itemize}

\end{frame}

\begin{frame}{Les combinaisons source / destination}

    Toutes les combinaisons ne sont pas autorisées.

    \begin{center}
        \begin{tabular}{|l|l|l|l|l|}
            \cline{2-5}
            \multicolumn{1}{l|}{~}
            & \multicolumn{1}{c|}{\textbf{src}}
            & \multicolumn{1}{c|}{\textbf{dest}}
            & \multicolumn{1}{c|}{\textbf{exemple}}
            & \multicolumn{1}{c|}{\textbf{en C}} \\
            \hhline{-====}

            \multirow{5}{*}{\texttt{movl}}
            & \multirow{2}{*}{Constante}
            & Registre
            & \texttt{movq \$0xA, \%rax}
            & %\texttt{tmp = 0xA;}
            \\
            \hhline{~~---}
            &
            & Mémoire
            & \texttt{movq \$0xB, (\%rax)}
            & %\texttt{*ptr = 0xB;}
            \\
            \hhline{~====}


            & \multirow{2}{*}{Registre}
            & Registre
            & \texttt{movq \$rax, \%rdx}
            & %\texttt{tmp1 = tmp2;}
            \\
            \hhline{~~---}
            &
            & Mémoire
            & \texttt{movq \$rdx, (\%rax)}
            & %\texttt{*ptr = tmp;}
            \\
            \hhline{~====}

            & Mémoire
            & Registre
            & \texttt{movq (\$rax), \%rdx}
            & %\texttt{tmp = *ptr;}
            \\
            \hline

        \end{tabular}
    \end{center}

\end{frame}

\begin{frame}
    \frametitle{Quand source et destination n'ont pas la même taille}

    Quand source et destination n'ont pas la même taille, deux variantes :

    \begin{itemize}
        \item \texttt{movs} complète la destination en copiant le bit de poids fort de la source,

        \item \texttt{movz} complète avec des 0.
    \end{itemize}
    
    \medskip

    Notes :
    \begin{itemize}
        \item marche seulement si la source est plus petite que la destination
        \item deux suffixes de taille sont utilisées, p.ex. \texttt{movzbl} ou \texttt{movslq}.
    \end{itemize}`
    
\end{frame}

\begin{frame}[fragile]
    \frametitle{Exemple}
   
    \begin{minted}{go}
func swap (p1 *int32, p2 *int64) {
    tmp1 := *p1 
    tmp2 := *p2 
    *p2 = int64(tmp1)
    *p1 = int32(tmp2)
}
    \end{minted}{go}

    \begin{center}
        \verb|go build -gcflags='-l'| puis \verb|objdump -d mov > mov.d|
    \end{center}

    \begin{minted}{gas}
main.swap:
  movslq (%rax),%rcx
  mov    (%rbx),%rdx
  mov    %rcx,(%rbx)
  mov    %edx,(%rax)
  ret
    \end{minted}{}
\end{frame}

\section{L'appel de procédures}

\begin{frame}{Appel de procédures, pile, cadre}

    Fonction/procédure du code source → procédure du code machine 
    \begin{itemize}
        \item attention, selon le langage, les deux objets peuvent avoir des noms $\neq$
    \end{itemize}

    \medskip 

    Appel et exécution de procédure dans le code machine utilisent \alert{la pile} :
    \begin{itemize}
    \item section dans les adresses hautes de l'espace d'adressage 
    \item croît vers les adresses basses (le sommet est en bas)
    \item adresse du sommet : {\tt \%rsp}
    \item reçoit les \alert{cadres} des procédures au fur et à mesure des appels
        \begin{itemize}
            \item Un \structure{cadre} (frame) reçoit les variables locales
            \item Début du cadre courant : {\tt \%rbp}
        \end{itemize}
    \end{itemize}

\end{frame}

\begin{frame}[fragile]{Pile et cadres (2)}

    \verb|main| appelle \verb|f1|, qui appelle \verb|f2|, qui appelle \verb|f3|. 

    \medskip 

\begin{verbatim}
 -------- <- base de la pile 
 | main |
 --------
 |  f1  |
 --------
 |  f2  |
 -------- <- %rbp (début cadre f3)
 |  f3  |
 -------- <- %rsp (sommet)
\end{verbatim}
    
    Une même procédure peut-elle avoir plusieurs cadres \og vivants \fg au même instant dans la pile ?
\end{frame}

\begin{frame}[fragile]{Principe général de l'appel de procédure}

    Dans l'ABI x86-64, l'appel de procédure repose sur deux opérations :

    Appel : \verb|call symbole|
    \begin{itemize}
        \item empile l'adresse de l'insruction suivante 
        \item charge \verb|%rip| avec l'adresse associée à \verb|symbole|
    \end{itemize}   

    \medskip

    Retour : \texttt{ret}
    \begin{itemize}
        \item dépile vers \verb|%rip|
        \item l'instruction exécutée juste après est donc celle dont l'adresse a été empilée par \verb|call|
    \end{itemize}

    \medskip 

    Pourquoi le retour de procédure utilise la pile plutôt qu'un « branchement » vers un symbole ?

\end{frame}

\begin{frame}[fragile]{Illustration : call}


\begin{verbatim}
47aec4:	e8 97 ff ff ff       	call   47ae60 <main.swap>
\end{verbatim}

\medskip

    {\footnotesize
\begin{verbatim}
...110  |            |            ...110 |            |
...108  |            |<- %rsp     ...108 |            |
                                  ...100 | 0x47aec9   |  
                                
        --------------                   --------------
   %rsp | ...108     |              %rsp | ...100     |  
        --------------                   --------------
   %rip | 0x47aec4   |              %rip | 0x47ae60   |
        --------------                   --------------
\end{verbatim}}

\end{frame}

\begin{frame}[fragile]
    \frametitle{Illustration : ret}

\begin{verbatim}
47ae6b:	c3                   	ret
\end{verbatim}

\medskip

    {\footnotesize
\begin{verbatim}
...110 |            |            ...110 |            |
...108 |            |            ...108 |            |<- %rsp 
...100 | 0x47aec9   |<- %rsp        

       --------------                   --------------
  %rsp | ...100     |              %rsp | ...108     |       
       --------------                   --------------
  %rip | 0x47ae6b   |              %rip | 0x47aec9   |
       --------------                   --------------
\end{verbatim}}

\end{frame}

\begin{frame}[fragile]{Allocation du cadre}

    Premières instructions d'une procédure (= prologue) :  allocation du cadre
    \begin{enumerate}
        \item empile la valeur courante de \verb|%rbp|
        \item fait pointer \verb|%rbp| sur le nouveau sommet de la pile 
        \item soustrait à \verb|%rsp| la taille du cadre à allouer (calculée par le compilateur)
    \end{enumerate}

    \medskip

    Dernières instructions (épilogue) : libération du cadre
    \begin{enumerate}
        \item ajoute à \verb|%rsp| la taille du cadre alloué
        \item dépile \verb|%rbp|
    \end{enumerate}

    \medskip

    Note : le code optimisé simplifie les choses en n'utilisant pas de pointeur de base.
\end{frame}

\begin{frame}[fragile]{Illustration : prologue}

\begin{verbatim}
080484c4 <main.swap>:
  47ae60:	55                   	push   %rbp
  47ae61:	48 89 e5             	mov    %rsp,%rbp
  47ae64:	48 83 ec 10          	sub    $0x10,%rsp
\end{verbatim}

\medskip

{\footnotesize
\begin{verbatim}
  ...108 |            |         ...108 |            | main.main 
  ...100 | 0x47af09   |<- %rsp  ...100 | 0x47af09   | 
                                ...0F8 | ...328     |<- %rbp
                                ...0F0 |            |  main.swap
                                ...0E8 |            |<- %rsp 
                            
         --------------               ---------------
    %rsp | ...1OO     |          %rsp | ...0E8      |
         --------------               ---------------
    %rbp | ...328     |          %rbp | ...0F8      |
         --------------               ---------------
    %rip | 0x47ae60   |          %rip | 0x47ae68    |
         --------------               ---------------
\end{verbatim}%
    } %
\end{frame}

\begin{frame}[fragile]{Illustration : épilogue}

\begin{verbatim}
  47ae9c:	48 83 c4 10          	add    $0x10,%rsp
  47aea0:	5d                   	pop    %rbp
  47aea1:	c3                   	ret
\end{verbatim}

{\footnotesize
    \begin{verbatim}
  ...108 |            |         ...108 |            |
  ...100 | 0x47af09   |         ...100 | 0x47af09   |<- %rsp
  ...0F8 | ...328     |<- %rbp         
  ...0F0 | 0x120      |                               
  ...0E8 |            |<- %rsp                         
                                                     
         --------------                --------------
    %rsp | ...0E8     |          %rsp  | ...100     |
         --------------                --------------
    %rbp | ...0F8     |           %rbp | ...328     |
         --------------                --------------
    %rip | 0x47ae9c   |           %rip | 0x47aea1   |
         --------------                --------------
 \end{verbatim}}

\end{frame}

\begin{frame}[fragile]
    \frametitle{Passage des paramètres et codes de retour}

    Dans l'ABI x86-64, les paramètre et les codes de retour sont passés
    \begin{itemize}
        \item via les registres, en commençant par \verb|%rax| puis \verb|%rbx| puis \ldots 
        \item au-delà de 10, ils sont passés par la pile
    \end{itemize}

    \medskip

    En plus de cette convention, plusieurs langage (dont go) assimilent les paramètres à des variables locales
    \begin{itemize}
        \item allouées par l'appelant
        \item mais initialisées et utilisées par l'appelé 
    \end{itemize}

    \medskip

    En go, les codes de retour sont aussi associés à des locales, allouées, initialisées et utilisées par l'appelé.

\end{frame}

\begin{frame}
    \frametitle{Sauvegarde et restauration des registres}

    L'appel de procédure ne doit pas interférer sur l'exécution de la fonction appelante → en dehors des codes de retour, l'état de la machine doit être restauré.

    \medskip

    Si l'appelant à des valeurs utiles dans les registres porteurs des codes de retour → empile avant l'appel, dépile (dans l'ordre inverse) après l'appel 

    \medskip

    Pour les autres registres, c'est à l'appelé de faire le travail : empile juste après le prologue, dépile (dans l'ordre inverse) juste après l'épilogue.

\end{frame}

\begin{frame}
    \frametitle{Optimisations}

    Un compilateur optimisant simplifie souvent beaucoup le protocole d'appel :

    \begin{itemize}
        \item pointeur de base inutile (le compilateur connaît la taille du cadre)
        \item allocations de locales pour les paramètres / codes de retour le plus souvent inutiles

        \item si aucune locale, aucun cadre alloué

        \item appel supprimé par \textit{inlining} des fonctions dans le code de l'appelant 

        \item instructions supprimées quand c'est possible par propagation des constantes
    \end{itemize}

    L'analyse d'un code optimisé nécessite d'avoir un esprit « ouvert »
\end{frame}

\section{Le registre de codes de condition et les instructions conditionnelles}

\begin{frame}
    \frametitle{Registre de code de conditions}

    Les unités fonctionnelles récoltent des informations sur les résultats.

    Par exemple, pour l'UAL :
    \begin{itemize}
        \item résultat nul
        \item retenue sortante 
        \item débordement
        \item parité
        \item \ldots
    \end{itemize}

    Ces informations sont accessibles depuis le registre de codes de condition (ou CCR pour \textit{Condition Code Register})

\end{frame}

\begin{frame}
    \frametitle{Le registre RFLAGS}

    Le CCR de l'ISA x86-64 se nomme \texttt{RFLAGS}.

    \medskip

    22 bits utilisés pour stocker des codes de condition (appelées \textit{flag}).

    \medskip

    Principaux flags : \texttt{CF}, \texttt{ZF}, \texttt{PF}, \texttt{OF}, \texttt{SF}
\end{frame}

\begin{frame}
    \frametitle{Mise à jour de RFLAGS}

    Les bits de \texttt{RFLAGS} sont mis à jour à chaque instruction arithmétique et logique.
    
    Les bits non touchés par une opération sont laissés à la valeur précédente.

    x86-64 propose également des opérations « sans destination » qui mettent à jour \texttt{RFLAGS}, par exemple : 
    \begin{itemize}
        \item \texttt{cmp a, b} : calcule $b - a$
        \item \texttt{test a, b} : calcule $a \land b$ (bit à bit)
        \item \texttt{bt r, n} : copie le nième bit de \texttt{r} dans \texttt{CF}
    \end{itemize}

\end{frame}

\begin{frame}
    \frametitle{Les conditions de l'ISA x86-64}

    Le jeu d'instruction propose un ensemble de conditions évaluées par lecture de RFLAGS

    \begin{center}
    \begin{tabular}{l|l|l}
        Symbole & Nom & Évaluation \\ 
        \hline 
        \texttt{a} & above & $\lnot (cf \lor zf)$ \\
        \texttt{b} & below & $cf$ \\
        \texttt{g} & greater & $\lnot(sf \oplus of) \land \lnot zf$ \\
        \texttt{l} & lower & $sf \oplus of$ \\
        \texttt{e}, \texttt{z} & equal, zero & $zf$ \\
        \texttt{ge} & greater or equal & $\lnot (sf \oplus of)$ \\
        \texttt{ne} & not equal & $\lnot zf$ \\
        \ldots & &  \\
    \end{tabular}
    \end{center}

    Quelle est la différence entre \texttt{a} et \texttt{g} ? Entre \texttt{b} et \texttt{l} ?

\end{frame}

\begin{frame}
    \frametitle{Les opérations conditionnelles}

    Les conditions sont utilisées par des opérations conditionnelles.

    \begin{block}{Opération conditionnelle}
        Opération dont le comportement dépend de la valeur d'une condition.
    \end{block}

    x86-64 propose 2 groupes d'opérations conditionnelles :
    \begin{itemize}
        \item \texttt{setxxx r} : $r ← 1$ si \texttt{xxx} et $0$ sinon.
        \item \texttt{jxxx adr} : saute à l'adresse \texttt{adr} si \texttt{xxx}, ne fait rien sinon.
        \item \texttt{cmovxxx a, b} : $b ← a$ si \texttt{xxx}.
    \end{itemize}

\end{frame}

\section{Les sauts et la traduction des structures de contrôle}

\begin{frame}
    \frametitle{Principe des sauts}

    Par défaut, le processeur exécute les instructions dans l'ordre où elles sont rangées en mémoire.
    
    \medskip

    Certaines instructions permettent de changer cet ordre, en modifiant arbitrairement la valeur du pointeur d'instruction.
    On parle de branchement, ou saut (vocabulaire x86-64).

    \medskip

    Déjà rencontrées : \texttt{call adr} et \texttt{ret}, utilisées pour l'appel de procédure.

    \medskip

    \texttt{jmp} et \texttt{jxxx}, où \texttt{xxx} est une condition.
    Utilisées pour traduire les structures de contrôle : conditionnelles, boucles.

\end{frame}
    
\begin{frame}[fragile]
    \frametitle{Traduction d'une conditionnelle simple}


    \begin{columns}[T]
        \begin{column}{.5\textwidth}
            {\small
            \begin{minted}{go}
func f1(a, b int) (r int) {
    if a >= b {
        r = a-b
    } else {
        r = b-a
    }
    return
}
            \end{minted}}
            Note : le code optimisé utilise \texttt{cmovxxx} et aucun saut. Pourquoi ?

        \end{column}
        \begin{column}{.5\textwidth}
            \scriptsize
            \begin{minted}{gas}
push   %rbp
mov    %rsp,%rbp
sub    $0x8,%rsp
mov    %rax,0x18(%rsp)
mov    %rbx,0x20(%rsp)
movq   $0x0,(%rsp)
cmp    %rbx,%rax
jge    47c5e2 
jmp    47c5eb 
sub    %rbx,%rax  # 47c5e2 
mov    %rax,(%rsp)
jmp    47c5f4 
sub    %rax,%rbx  # 47c5eb
mov    %rbx,(%rsp)
jmp    47c5f4 
mov    (%rsp),%rax # 47c5f4
add    $0x8,%rsp
pop    %rbp
ret
           \end{minted} 
        \end{column}
    \end{columns}
\end{frame}

\begin{frame}[fragile]
    \frametitle{Traduction d'une boucle simple}
    \begin{columns}[T]
        \begin{column}{.5\textwidth}
            {\small
            \begin{minted}{go}
func f2(x  int) (r int) {
    r = 1
    for cpt := 1; cpt < x; 
                cpt+=2 {
        r *= cpt
    }
    return
}
            \end{minted}}
        \end{column}
        \begin{column}{.5\textwidth}
            \scriptsize
            \begin{minted}{gas}
mov    $0x1,%ecx
mov    $0x1,%edx
jmp    47c5f7 
lea    0x2(%rcx),%rbx  # 47c5ec
imul   %rcx,%rdx
mov    %rbx,%rcx
cmp    %rcx,%rax  # 47c5f7
jg     47c5ec  
mov    %rdx,%rax
nop
ret
           \end{minted} 
        \end{column}
    \end{columns}
    
\end{frame}


\begin{frame}[fragile]
    \frametitle{Traduction d'une conditionnelle complexe}

    \begin{columns}
        \begin{column}{.5\textwidth}
    \small
    \begin{minted}{go}
func f3(x int) (r int) {
    x = x % 6
    switch(x) {
    case 0:
        r = 2*x
    case 1:
        r = 1/x
    case 2:
        r = x*x
    case 3:
        r = x/2
    case 4:
        r = x+3
    case 5:
        r = x-7
    }
    return
}
    \end{minted}
       \end{column} 
        \begin{column}{.5\textwidth}
            Cf. fichier.
        \end{column}
    \end{columns}

\end{frame}

\section{Bilan et conclusions}

\begin{frame}{Bilan}
  
    Ce qu'on a vu et qui doit maintenant être connu :
    \begin{itemize}
        \item Rappel sur la compilation et la création d'exécutable
        \item Principes de base de l'organisation d'un ordinateur 
        \item Zoom sur l'unité centrale, notion de registre, d'instruction, d'opérations
        \item Organisation de la mémoire d'un processus (sections)
        \item Faire le lien entre code source et programme binaire 
    \end{itemize}

\end{frame}

\begin{frame}
    \frametitle{Conclusions}

    Quelques leçons pour la suite de vos études et votre carrière :
    \begin{itemize}
        \item Pour optimiser votre code, en premier lieu, faites confiance au compilateur
        \item Pour bien comprendre la sémantique des langages, il faut penser à la traduction en code machine 
        \item L'étude approfondie des architectures matérielles et de l'interface matérielle-logicielle est au centre de plusieurs sujets :
            \begin{itemize}
                \item performances « avancées »
                \item cybersécurité
                \item systèmes d'exploitation 
                \item informatique temps réel
                \item \ldots
            \end{itemize}
    \end{itemize}
   
    Si vous souhaitez approfondir le sujet, il y a quelques ouvrages recommandables à la bibliothèque.
\end{frame}

%\begin{frame}{Allocation du cadre (4)}

    %Dès le niveau d'optimisation 1, \texttt{\%ebp} n'est plus utilisé pour mémoriser le début du cadre

    %\medskip
    
    %Allocation et désallocation se font en manipulant \texttt{\%esp} ( à l'aide des instructions \texttt{add} et \texttt{sub}) 

    %\medskip
    
    %La taille du cadre est codée en dur dans le code engendré

    %\medskip

    %Dans certains cas, la taille du cadre change au fur et à mesure de l'exécution de la fonction, selon les besoins

%\end{frame}

%\begin{frame}{Passage de paramètre et code de retour}

    %Les paramètres sont empilés par la procédure appelante juste avant l'appel.

    %\medskip

    %En C, les paramètres sont évalués puis empilés de la droite vers la gauche.

    %\begin{itemize}
        %\item Pour du code non optimisé, les paramètres sont accédés via \texttt{\%ebp}
            %\begin{itemize}
                %\item Après la préparation du cadre, le paramètre de gauche est à l'adresse \texttt{8(\%ebp)}
            %\end{itemize}
        %\item Pour du code optimisé, les paramètres sont accédés via \texttt{\%esp}
            %\begin{itemize}
                %\item Après la préparation du cadre, le paramètre de gauche est à l'adresse \texttt{c(\%esp)}, où \texttt{c} est égal à la taille du cadre + 4
            %\end{itemize}

    %\end{itemize}
   
    %Par convention, une procédure stocke son code de retour dans le registre \texttt{\%eax}
%\end{frame}

%\begin{frame}{Sauvegarde des registres}

    %Appelant et appelé ont chacun leur cadre mais partagent les mêmes registres
    
    %\medskip

    %\begin{itemize}
        %\item l'appelant peut vouloir récupérer les registres dans l'état précédent l'appel
        %\item l'appelé peut vouloir utiliser les registres
    %\end{itemize}

    %\medskip

    %Qui doit assurer la sauvegarde (en empilant) et la restauration (en dépilant) ? 
    %L'ABI IA32 coupe la poire en 2 :
    %\begin{itemize}
        %\item l'appelant pour : \texttt{\%eax}, \texttt{\%edx}, \texttt{\%ecx}
        %\item l'appelé pour : \texttt{\%ebx}, \texttt{\%esi}, \texttt{\%edi}
    %\end{itemize}

%\end{frame}

%\begin{frame}{Récapitulatif}

    %\begin{columns}
        %\begin{column}{0.5\textwidth}
            %Appelant :
            %\begin{enumerate}
                %\item construire et empiler les paramètres*
                %\item sauver les registres*
                %\item appeler la procédure
                    %\setcounter{enumi}{9}
                %\item sauver le résultat*
                %\item restaurer les registres*
            %\end{enumerate}
        %\end{column}
        %\begin{column}{0.5\textwidth}
            %Appelé :
            %\begin{enumerate}
                %\setcounter{enumi}{3}
                %\item sauver les registres*
                %\item allouer un cadre*
                %\item exécuter le code de la procédure
                %\item désallouer le cadre*
                %\item restaurer les registres*
                %\item retourner
            %\end{enumerate}
        %\end{column}
    %\end{columns}

    %\bigskip

    %Les étapes marquées d'une * sont optionnelles
%\end{frame}

%\begin{frame}[fragile]{Exemples}

%Swap, sans optimisation :
%\begin{verbatim}
%$ gcc -fno-pie -fno-pic -no-pie -m32 -ggdb swap.c
%$ objdump -d a.out > swap0.d
%\end{verbatim}

%\medskip

%Swap, avec optimisation :
%\begin{verbatim}
%$ gcc -fno-pie -fno-pic -no-pie -m32 -ggdb -O1 swap.c
%$ objdump -d a.out > swap1.d
%\end{verbatim}

%\medskip


%\end{frame}


%\section{Les branchements et les structures de contrôle}

\end{document}
